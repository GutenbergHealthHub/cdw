\chapter{\acs{kis}} \label{chp:kis}

Hier werden die tagesaktuellen extrahierten Daten zu Patienten, Fällen, Bewegungen, Diagnosen und Prozeduren direkt aus dem Quellsystem \ac{kis} gespeichert. Mit Hilfe diesem Schema lassen sich viele der Abbildungen für weitere Projekte realisieren.

In diesem Schema behalten die Tabellen denselben Namen wie in \ac{kis} (Tabelle \ref{tab:schemaKis}). Die Dokumentation der Tabellen in \ac{kis} befinden in der Confluence Seite von Medizin Informatik der Universitätsmedizin Mainz.

\begin{table}[ht]
	\centering   
	\caption{Tabellen im Schema \acs{kis}}
	\label{tab:schemaKis}
	\begin{tabular}{||l|l||}   		
		\hline
		View & Beschreibung \\ [0.5ex]
		\hline\hline
		\texttt{nbew} & Bewegungen \\
		\hline
		\texttt{ndia} & Diagnosen \\
		\hline
		\texttt{nfal} & Fälle \\
		\hline
		\texttt{nicp} & Prozeduren \\
		\hline
		\texttt{npat} & Patienten \\
		\hline
		\texttt{norg} & Organisationseinheiten \\
		\hline
		\texttt{screencov} & COVID-19 \\
		\hline
	\end{tabular}
\end{table}

\section{Bewegungen} \label{sec:beweg}

Die Bewegungen der Fälle während des Hospitalisieren.

\subsection{Behandlungskategorie} \label{subsec:behKat}

\subsubsection{Insgesamt} \label{subsubsec:behKatI}

\begin{table}[ht]
	\centering   
	\caption{View dqa\_nbew\_bkat}
	\label{tab:beweBkatAll}
	\begin{tabular}{||l|l|p{10cm}||}   		
		\hline
		Spalte & Datentyp & Beschreibung \\ [0.5ex]
		\hline\hline
		quantity & bigint & Menge an Fälle mit einer bestimmten Behandlungskategorie in der Tabelle \texttt{nbew} \\
		\hline
		bkat & varchar & Id der Behandlungskategorie (NULL bei nicht existierender Behandlungskategorie)\\
		\hline
		behandlungskategorie & varchar & Behandlungskategorie (NULL bei nicht existierender Behandlungskategorie)\\
		\hline
		
	\end{tabular}
\end{table}

\subsubsection{Jährlich} \label{subsubsec:behKatJ}

\begin{table}[ht]
	\centering   
	\caption{View dqa\_nbew\_bkat\_jahr}
	\label{tab:beweBkatJahr}
	\begin{tabular}{||l|l|p{10cm}||}   		
		\hline
		Spalte & Datentyp & Beschreibung \\ [0.5ex]
		\hline\hline
		quantity & bigint & Menge an Fälle mit einer bestimmten Behandlungskategorie in der Tabelle \texttt{nbew} \\
		\hline
		bkat & varchar & Id der Behandlungskategorie (NULL bei nicht existierender Behandlungskategorie)\\
		\hline
		behandlungskategorie & varchar & Behandlungskategorie (NULL bei nicht existierender Behandlungskategorie)\\
		\hline
		jahr & int &  Jahr des Datum der Bewegung \\
		\hline
		
	\end{tabular}
\end{table}


\subsection{Betten Belegung} \label{subsec:bett}

\subsubsection{Insgesamt} \label{subsubsec:bettI}

\begin{table}[ht]
	\centering   
	\caption{View dqa\_nbew\_bett}
	\label{tab:beweBettAll}
	\begin{tabular}{||l|l|p{10cm}||}   		
		\hline
		Spalte & Datentyp & Beschreibung \\ [0.5ex]
		\hline\hline
		quantity & bigint & Menge an Fälle an bestimmten Betten in der Tabelle \texttt{nbew} \\
		\hline
		bett & varchar & Id des Bettes (NULL bei nicht existierenden Bett)\\
		\hline
		
	\end{tabular}
\end{table}

\subsubsection{Jährlich} \label{subsubsec:bettJ}

\begin{table}[ht]
	\centering   
	\caption{View dqa\_nbew\_bett\_jahr}
	\label{tab:beweBettJahr}
	\begin{tabular}{||l|l|p{10cm}||}   		
		\hline
		Spalte & Datentyp & Beschreibung \\ [0.5ex]
		\hline\hline
		quantity & bigint & Menge an Fälle an bestimmten Betten in der Tabelle \texttt{nbew} \\
		\hline
		bett & varchar &  Id des Bettes (NULL bei nicht existierenden Bett) \\
		\hline
		jahr & int &  Jahr des Datum der Bewegung \\
		\hline
		
	\end{tabular}
\end{table}

