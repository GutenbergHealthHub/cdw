\chapter{Nutzung}
	   \section{Start die \acs{db}}
	   Die Instanz startet automatisch nach jedem Reboot des Server. Wenn die Instanz auf diese Weise nicht startet, sollte man folgendes machen / überprüfen:
	   \begin{itemize}
	   	\item \texttt{ssh IP des Server -l cdw} \# Login auf dem Server via ssh mit einem Benutzer mit sudo Rechten, unter Windows auch mit den Tools putty oder MobaXterm
	   	\item Die Partition der Instanz sollte automatisch gemountet werden, da es in fstab konfiguriert ist. Falls die Partition nicht gemountet ist, sollte man die folgende Schritte durchführen:
	   	\begin{itemize}
	   	 \item cdw\$ \texttt{lsblk} \# Überprüfen ob die Partition /dev/sdb1 gemountet ist.
	   	 \item cdw\$ \texttt{sudo mount /dev/sdb1 /media/db} \#Falls die Partition /dev/sdb1 nicht gemountet ist
	   	\end{itemize}	   	  
	   	\item Die PostgreSQL-Instanz startet automatisch nach 100 Sekunden nach jedem Neu Start des Servers, da das Script zum Starten via cron-daemon abgerufen wird:
	   	\begin{itemize}
	   		\item Befehl in crontab: 
	   		
	   		\texttt{@reboot sleep 100 \&\& /media/db/cdw\_database/startDB.sh}
	   	\end{itemize}
	   	Falls die Instanz nicht automatisch startet sollte man diese Befehlen verfolgen.
	   	\begin{itemize}
	   		\item cdw\$ \texttt{sudo su - clinicuser} \#Benutzer ändern
	   		\item clinicuser\$ \texttt{cd /media/db/database} \# Gehe zum Ordner der  Instanz
	   	    \item clinicuser\$ \texttt{/usr/local/pg12tde/bin/pg\_ctl -D clinic\_instance  restart} \#(Re)Start die Instanz
	   	\end{itemize}
	   \end{itemize}
   \section{Arbeiten mit \acs{csdwh} via ssh/psql}
   
   \begin{itemize}
   	\item \texttt{ssh Server\_IP -l tooluser} \# Login auf dem Server
   	\item tooluser\$ \texttt{/usr/local/pg12tde/bin/psql -p 5433 database\_name -U user\_name} \#Verbindung mit einer Datenbank der Instanz 
   	\item database\_name\#\texttt{\textbackslash c another\_database\_name} -- Verbindung mit anderer \ac{db}
   \end{itemize}
