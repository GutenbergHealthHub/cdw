\chapter{Schemata}
\label{ch: schema}

Die Schemata speichern die "rohe" \textbf{pseudonymisierte} Information der ursprünglichen Systems oder die Metadaten. Diese Daten werden in Views analysiert oder weiter verarbeitet für andere Anwendungen oder Projekten. Wichtige Hinweis ist, dass die Daten in dem Data Warehouse unverändert bleiben sollen.

Die Dokumentation der Views für Datenqualität liegt in einem anderem Dokument.
  \section{p21}
  Dieses Schema speichert die jährliche Information der \S21, die von Medizincontrolling in CSV-Dateien generiert wird. 
  
  Der jährliche Rhythmus ist zu groß, als dass die Daten bspw. zur Rekrutierung von Patienten für Studien aber auch zu Forschung genutzt werden können. Auf diesem Grund wird diese Information in der Zukunft nicht mehr aus CSV-Dateien genommen sondern direkt aus dem \ac{kis}.

  \subsection{Tabellen}
  \begin{table}[ht]
  	\centering   
  	\caption{Tabellen im Schema p21}
  	\begin{tabular}{||l|l||}
  		\hline
  		Tabelle & Beschreibung \\ [0.5ex]
  		\hline\hline
  		\texttt{p21\_encounter} & Information der Datei FALL.csv: Fälle \\
  		\hline
  		\texttt{p21\_department} & Inhalt der Datei  FAB.csv: Fachabteilung \\
  		\hline
  		\texttt{p21\_operation} & Information der Datei  OPS.csv: Operationen \\
  		\hline
  		\texttt{p21\_diagnosis} & Basiert auf der Datei ICD.csv: Diagnosen (\ac{icd10gm}) \\
        \hline
  	\end{tabular}
  \end{table}
  
\subsection{Views für Tools}
In diesem Schema befinden sich auch die Views für \ac{etl}-Prozessen die, solche Information aus \S21 benötigen. Der Inhalt dieser Views entspricht die Formatierung der \ac{khentgg} \S 21 Übermittlung und Nutzung der Daten.
     	\begin{table}[ht]
     	\centering   
    	\caption{Views im Schema p21}
     	\begin{tabular}{||l|l||}
     	
     		\hline
     		View & Beschreibung \\ [0.5ex]
     		\hline\hline
     		\texttt{fall} & Falldaten \\
     		\hline
     		\texttt{fab} & Fachabteilungsangaben \\
     		\hline
     		\texttt{icd} & Diagnosenangaben \\
     		\hline
     		\texttt{ops} & Prozedurenangaben \\
     		\hline
     	\end{tabular}
    \end{table}

  \section{kis}
   Hier werden die tagesaktuellen extrahierten Daten zu Patienten, Fällen, Bewegungen, Diagnosen und Prozeduren direkt aus dem Quellsystem \ac{kis} gespeichert. Mit Hilfe diesem Schema lassen sich viele der Abbildungen für weitere Projekte realisieren.
  
   \subsection{Tabellen}
  In diesem Schema behalten die Tabellen denselben Namen wie in \ac{kis}. 
   \begin{table}[ht]
   	\centering   
   	\caption{Tabellen im Schema kis}
   	\begin{tabular}{||l|l||}   		
   		\hline
   		View & Beschreibung \\ [0.5ex]
   		\hline\hline
   		\texttt{nbew} & Bewegung \\
   		\hline
   		\texttt{ndia} & Diagnosen \\
   		\hline
   		\texttt{nfal} & Fälle \\
   		\hline
   		\texttt{nicp} & Prozeduren \\
   		\hline
   		\texttt{npat} & Patienten \\
   		\hline
   		\texttt{norg} & Organisationseinheiten \\
   		\hline
   	\end{tabular}
   \end{table}
  
  \section{\ac{pdms}}
  Hier wird die tagesaktuelle Information aus dem COPRA-System gespeichert. Dieses Schema beinhaltet Befunde, ärztliche Anweisungen und Überblick über Behandlungsschritte.
  \subsection{Tabellen}
  In diesem Schema behalten die Tabellen denselben Namen wie im COPRA-System. 
  \begin{table}[ht]
  	\centering   
  	\caption{Tabellen im Schema copra}
  	\begin{tabular}{||l|l||}   		
  		\hline
  		Tabelle & Beschreibung \\ [0.5ex]
  		\hline\hline
  		\texttt{co6\_data\_decimal\_6\_3} & Metadaten der nummerischen Messungen \\
  		\hline
  		\texttt{co6\_data\_object} & Metadaten der Messungen von Typ Objekt\\
  		\hline
  		\texttt{co6\_medic\_data\_patient} & Demografische Information der Patienten \\
  		\hline
  		\texttt{co6\_medic\_pressure} & Daten der Herz-Messungen\\
  		\hline
  		\texttt{co6\_config\_variables} & Variables in copra\\ \hline
  		\texttt{co6\_config\_variable\_types} & Art der Variables in copra\\ \hline
  	\end{tabular}
  \end{table}

  \section{gtds} 
  Dieses Schema speichert die Daten der mainzenen Instanz des \ac{gtds} und somit die Erfassung und Verarbeitung der Daten der revidierten Basisdokumentation klinischen Krebsregistern.
  \subsection{Tabellen}
  Dieses Schema hat momentan nur eine Tabelle. Die ist auf eine View auf eine Auswertung auf die Daten des \ac{gtds} basiert.
  \begin{table}[ht]
  	\centering   
  	\caption{Tabellen im Schema gtds}
  	\begin{tabular}{||l|l||}   		
  		\hline
  		Tabelle & Beschreibung \\ [0.5ex]
  		\hline\hline
  		\texttt{auswertung\_diz} & Auswertung auf Daten auf \ac{gtds} \\
  		\hline
  	\end{tabular}
  \end{table}
  \section{centrallab}
  Hier werden die Daten aus dem Zentrallabor (Institut für Klinische Chemie und Laboratoriumsmedizin) gespeichert.
  \subsection{Tabellen}
  Die Tabellen speichern die Messungen sowie Mapping zu LOINC-Code.
  \begin{table}[ht]
  	\centering
  	\caption{Tabellen im Schema centrallabor}
  	\begin{tabular}{|| l | l ||}
  		\hline
  		Tabelle & Beschreibung \\[0.5ex]
  		\hline\hline
  		\texttt{observation} & Laborwerte der Patienten \\
  		\hline
  		\texttt{observationreport} & Verlinkung der Laborwerten mit Fälle und Patienten \\
  		\hline
  		\texttt{loinc\_mapping\_central\_lab} & Mapping der LOINC-Code zu der Messungen und/Geräte \\
  		\hline
  	\end{tabular}
  \end{table}

\section{imagic}
Hier wird die Information aus dem IMAGIC-System gespeichert. Dieses Schema beinhaltet Information aus der Hautklinik. Davon Metadaten der Bilder sowie Befunde anhand der Bilder.
\subsection{Tabellen}
In diesem Schema behalten die Tabellen denselben Namen wie im IMAGIC-System. 
\begin{table}[ht]
	\centering   
	\caption{Tabellen im Schema imagic}
	\label{tb:ima}
	\begin{tabular}{||l|l||}   		
		\hline
		Tabelle & Beschreibung \\ [0.5ex]
		\hline\hline
		\texttt{image} & Metadaten der Bilder \\
		\hline
		\texttt{patient} & Patienten Informationen \\
		\hline
		\texttt{study} & Information der Studien an der Hautklinik \\
		\hline
		\texttt{visit} &  Besuch/Fall-Information\\
		\hline
	\end{tabular}
\end{table}

  \section{metadata\_repository}
  Dieses Schema speichert die Information der Metadaten aller anderen Schemata.
  
  \subsection{Tabellen}
  \label{tb:metarepo}
  \begin{longtable}{||p{5cm}|p{7cm}|p{4cm}||}  	
  	\caption{Tabellen in Schema metadata\_repository} \\  	
  	\hline Table & Beschreibung & Schemata/Quellen\\ \hline \hline
  	abschluss\_grund & Auswahlliste - Abschluss Grund & \ac{gtds} \\ \hline
  	ann\_arbor\_allgemein & Ann-Arbor Allgemeinsymptomatik & \ac{gtds} \\ \hline
  	ann\_arbor\_extra & Ann-Arbor extralymphatischer Befall &  \ac{gtds} \\ \hline
  	ann\_arbor\_stadium & Auswahlliste - Ann-Arbor Stadium & \ac{gtds} \\ \hline
  	applikationsart & Applikationsart der (Teil-)Bestrahlung & \ac{gtds} \\ \hline
  	applikationstechnik & Applikationstechnik der (Teil-)Bestrahlung & \ac{gtds} \\ \hline
  	arzt\_anlass & Auswahlliste - Arzt Anlass & \ac{gtds}\\ \hline
  	autopsie & Autopsie & \ac{gtds} \\ \hline
  	behandlungsanlass & Auswahlliste - Tumor Ausprägung & \ac{gtds} \\ \hline
  	behandlungskategorie & Behandlungskategorie & \ac{kis} \\ \hline
  	bewegunsart & Bewegunsart  & \ac{kis} \\ \hline
  	bewegungstyp & Bewegungstyp  & \ac{kis} \\ \hline
  	body\_localisation & Lokalisation im Körper & \S 21 \ac{khentgg} \\ \hline
  	complication\_level & Komplication Ebene & \ac{kis} \\ \hline
  	copra\_biosignal & Biosignalen & \ac{pdms} \\ \hline
  	copra\_variables & Konfiguration Variablen & \ac{pdms} \\ \hline
  	country & Länder & \ac{kis} \\ \hline
  	todesursache & Todesursache & \ac{kis} \\ \hline
  	fachabteilung & Fachabteilung & \ac{kis}, \S 301 Abs. 3 \acs{sgb} V, \S 21 \ac{khentgg} \\ \hline
  	fachabteilung\_prefix & Präfix der Fachabteilung  & \S 301 Abs. 3 \acs{sgb} V, \S 21 \ac{khentgg}  \\ \hline
  	diagnosenzusatz & Diagnosenzusatz & \ac{kis} \\ \hline
  	diag\_gewissheit & Diagnostische Gewissheit & \ac{kis} \\ \hline
  	diagnosis\_type & Art der Diagnose & \ac{kis}, \S 21 \ac{khentgg} \\ \hline
  	diagnosis\_type\_icd & Art der \acs{icd10gm}-Diagnose & \ac{kis}, \acs{bfarm} \\ \hline
  	diagsich\_hoechste & Höchste Diagnosesicherung & \ac{gtds} \\ \hline
  	drg\_prozedur & Kategorie einer DRG-Prozedur & \ac{kis} \\ \hline
  	einrichtung & Einrichtung & \ac{kis} \\ \hline
  	einweisungs\_ueberweisungs \_nachbehandlungsart & Einweisungs-, Überweisungs,- Nachbehandlungsart & \ac{kis}\\ \hline
  	einwilligung & Einwilligung & \ac{gtds}\\ \hline
  	erfassungsanlass & Erfassungsanlass & \ac{gtds} \\ \hline
  	fachrichtung & Fachrichtung & \ac{kis} \\ \hline
  	fallendes & Art des Fallendes & \ac{kis} \\ \hline
  	fallstatus & Fallstatus & \ac{kis} \\ \hline
  	falltyp & Falltyp & \ac{kis} \\ \hline
  	gender & Geschlecht & \ac{kis}, \ac{gtds}, \ac{pdms}, \S 21 \ac{khentgg} \\ \hline
  	gtds\_datenart & \ac{gtds}-Datenart & \ac{gtds} \\ \hline
  	herkunft & Herkunft & \ac{gtds} \\ \hline
  	histo\_diagnose & Genkennzeichnung der Auswertungsrelevanz der Histologie & \ac{gtds} \\ \hline
  	histo\_grading & Histologisches Grad & \ac{gtds} \\ \hline
  	histo\_haupt\_neben & Haupt- oder Nebenhistologie & \ac{gtds} \\ \hline
  	inter\_status\_amb\_besu & Interner Status eines ambulanten Besuchs & \ac{kis} \\ \hline
  	lok\_haupt\_neben & Kennzeichnung ob Haupt- oder Nebenlokalisation & \ac{gtds} \\ \hline
  	lok\_rezidivart & Art des Rezidives & \ac{gtds} \\ \hline
  	lok\_seite & Seitenangabe zur Lokalisation & \ac{gtds} \\ \hline
  	lokalisation & Lokalisationsschlüssel & \ac{gtds} \\ \hline
  	op\_intention & Intention der Operation & \ac{gtds} \\ \hline
  	operation\_katalog & Identifikation eines Operationsleistungskataloges & \ac{kis} \\ \hline
  	orgtype & Typ der Organisationseinheit & \ac{kis} \\ \hline
  	p21\_admission\_cause & Aufnahmeanlass & \S 301 Abs. 3 \acs{sgb} V, \S 21 \ac{khentgg} \\ \hline
  	p21\_admission\_reason & Aufnahmegrund & \S 301 Abs. 3 \acs{sgb} V, \S 21 \ac{khentgg} \\ \hline
  	p21\_admission\_reason\_1\_2 & Aufnahmegrund 1. und 2. Stellen & \S 301 Abs. 3 \acs{sgb} V, \S 21 \ac{khentgg}\\ \hline
  	p21\_admission\_reason\_3 & Aufnahmegrund 3. Stelle & \S 301 Abs. 3 \acs{sgb} V, \S 21 \ac{khentgg} \\ \hline
  	p21\_admission\_reason\_4 & Aufnahmegrund 4. Stelle & \S 301 Abs. 3 \acs{sgb} V, \S 21 \ac{khentgg} \\ \hline
  	p21\_department & Fachabteilung & \S 301 Abs. 3 \acs{sgb} V, \S 21 \ac{khentgg} \\ \hline
  	p21\_department\_code\_mean & Bedeutung der Ziffer im Code der Fachabteilungen & \S 301 Abs. 3 \acs{sgb} V, \S 21 \ac{khentgg} \\ \hline
   	p21\_merging\_reason & Fallzusammenführungsgrund & \S 301 Abs. 3 \acs{sgb} V, \S 21 \ac{khentgg} \\ \hline
  	p21\_remission\_reason & Entlassungs-/Verlegungsgrund & \S 301 Abs. 3 \acs{sgb} V, \S 21 \ac{khentgg} \\ \hline
  	p21\_remission\_reason\_1\_2 & Entlassungs-/Verlegungsgrund 1. und 2. Stellen & \S 301 Abs. 3 \acs{sgb} V, \S 21 \ac{khentgg} \\ \hline
  	p21\_remission\_reason\_3 & Entlassungs-/Verlegungsgrund 3. Stelle & \S 301 Abs. 3 \acs{sgb} V, \S 21 \ac{khentgg} \\ \hline
  	p21\_remuneration\_area & Entgeltbereich & \S 301 Abs. 3 \acs{sgb} V, \S 21 \ac{khentgg} \\ \hline
  	protokoll\_typ & Protokoll-Typ & \ac{gtds}\\ \hline
  	prozedur\_typ & Prozedurtyp & \ac{kis}\\ \hline
  	pupille\_reaction & Reaktion der Pupille & \ac{pdms}, \acsu{aktin}-Projekt \\ \hline
  	pupille\_width & Erweiterung der Pupille & \ac{pdms}, \acs{aktin}-Projekt \\ \hline
  	r\_klassifikation & R Klassifikation & \ac{gtds} \\ \hline
  	rezidiv & Rezidiv & \ac{gtds}\\ \hline
  	sources & Information der System- Datenquellen & System- Datenquellen\\ \hline
  	status\_color & Status Farbe & \ac{pdms}, \acs{aktin}-Projekt \\ \hline
  	status\_datenart & \ac{gtds}-Datenart zu einer Status & \ac{gtds} \\ \hline
  	sterbedatum\_exakt & sterbedatum\_exakt & \\ \hline
  	stornierungsgrund & Stornierungsgrund & \ac{kis}\\ \hline
  	strahlenart & Strahlenart & \ac{gtds} \\ \hline
  	tn24 & Tabelle TN24 in \ac{kis} (\acsu{ish}: Behandlungskategorien) & \ac{kis} \\ \hline
  	transportart & Transportart & \ac{kis} \\ \hline
  	tumorauspraegung & Tumorausprägung & \ac{gtds} \\ \hline
  	tumortod & Tumortod & \ac{gtds} \\ \hline
  	unfallart & Unfallart & \ac{kis} \\ \hline
  	units & Einheiten & \ac{kis} \\ \hline
  	zipcode & Postleihzahl & \ac{kis} \\ \hline  
  \end{longtable}