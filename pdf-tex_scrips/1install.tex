\chapter{Installation und Konfiguration der Instanz des \acs{csdwh}}
    
    Das \ac{csdwh} für die Forschung Daten befindet sich in einem Ubuntu Server mit der Version Ubuntu 18.04 \ac{lts}. Diese \ac{db} wurde in PostgreSQL mit Hilfe von PostgreSQL \ac{tde} implementiert und verschlüsselt. Das bedeutet, dass alle Dateien aus der PostgreSQL-Instanz sind auf dem Festplatte verschlüsselt gespeichert und bei lesen entschlüsselt.
    
    \section{\acs{tde}-Installation}    
    
    Die Installation von PostgreSQL \ac{tde} Version postgresql-12.3\_TDE\_1.0 verfolgte die Installation Guide des Software unter dem Link  (\url{https://www.cybertec-postgresql.com/de/transparent-data-encryption-installation-guide/}). Davor waren die notwendige Paketen in dem Ubuntu-Server installiert via \textsf{apt}:
    
    \begin{tabular}{ll}
    	\textsf{zlib1g-dev} & \textsf{libssl-dev} \\ 
    	\textsf{libldb-dev} & \textsf{libldap2-dev} \\
    	\textsf{libperl-dev} & \textsf{python-dev} \\
    	\textsf{libreadline-dev} & \textsf{libxml2-dev} \\
    	\textsf{libxslt1-dev} & \textsf{bison} \\
    	\textsf{flex} & \textsf{uuid-dev} \\
    	\textsf{make} & \textsf{make} \\
    	\textsf{gcc} & \textsf{libsystemd-dev} \\
    	\textsf{libxml2-utils} & \textsf{xsltproc} \\
    \end{tabular} \\
    
    Der Install-Kommando lautet: 
           
    sudo ./configure --prefix=/usr/local/pg12tde --with-openssl --with-perl --with-python --with-ldap --with-libxml --with-uuid=e2fs --with-systemd\\ \\
    
    Der Start-Kommando für die Instanz lautet: 
    
    /usr/local/pg12tde/bin/initdb -D /media/db/cdw\_database/clinic\_instance
    

	\section{\acs{tde}-Konfiguration} 
	
    \subsection{\acs{tde}-Instanz-Dateien}
    Auf dem Betriebssystem wurde den Benutzer clinicuser angelegt, der dient die Administration der \ac{db}-Instanz und besitzt keine administrative rechte auf dem Betriebssystem.
    Die Dateien der \ac{tde}-Instanz befinden sich in dem Server unter \textsf{/media/db/cdw\_database}.
    \begin{itemize}
    	\item \textsf{clinic\_instance} -- Instanz der \ac{csdwh} mit \ac{db}- und Konfigurationsdateien.
    	%\item dbkeys -- Ordner der mit Schlüssel-Datei. Jede Schlüssel in der Datei hat 1000 Zufall-Zeichenketten, jede davon ist 1000 Charakter lang mit Buchstaben, Zahlen und Sonderzeichen, davon wird eine genommen, verschlüsselt und als Schlüssel für dem Start des PostgreSQL-\ac{TDE}-Services benützt.
    	\item \textsf{sh\_scripts} -- Shell-Skripts.
    	\begin{itemize}
           \item \textsf{clinic\_instance\_key.sh} -- Skript fürs Schlüssel-Manager. Der Schlüssel der Instanz ist ein \ac{md5}-Hash.
    	\end{itemize}
    	\item \textsf{dbBack} -- Täglicher Backup der ganzen Instanz. Hier werden die fünf letzte Backups der \ac{db} in verschlüsselten \ac{zip}-Dateien aufbewahren. Die Nomenklatur der Dateien ist \textsf{staging\_YYYY-MM-TT.all.zip}.
    \end{itemize}

    \subsection{Konfigurationsdateien}
    Die Datei \textsf{postgresql.conf} wurde wie folgt modifiziert:
    \begin{itemize}
    	\item \textsf{port = 5433} \#Proxy der Instanz
    	\item \textsf{listen\_addresses = '*'} \# Maschinen auf denen die Instanz abrufbar ist
    	\item \textsf{password\_encryption = scram-sha-256} \# Kennwort-Verschlüsselung Protokoll
    	\item \textsf{encryption\_key\_command = '/media/db/cdw\_database/sh\_scripts/clinic\_instance\_key.sh'} \# Datei mit dem Schlüssel der Instanz
    \end{itemize}

	In der Datei \textsf{pg\_hba.conf} wurden die Benutzer der Instanz definiert.
	\begin{itemize}
		\item \textsf{local   all     all                      scram-sha-256} \# lokale Verbindungen
		\item \textsf{host    all             all              0.0.0.0/0         scram-sha-256} \# externe Verbindungen
	\end{itemize}
