\chapter{Installation und Konfiguration der Instanz des \ac{cDWH}}
    
    Die PostgreSQL-Instanz wurde mit Hilfe von PostgreSQL \ac{TDE} verschlüsselt. Das bedeutet, dass alle Dateien aus der PostgreSQL-Instanz sind auf dem Festplatte verschlüsselt gespeichert und bei lesen entschlüsselt.
    \section{\ac{TDE}-Installion}    
    Die Installation von PostgreSQL \ac{TDE} Version postgresql-12.3\_TDE\_1.0 verfolgte die Installation Guide des Software (\url{https://www.cybertec-postgresql.com/de/transparent-data-encryption-installation-guide/}) Davor waren die notwendige Bibliotheken und Software installiert:
    \begin{itemize}
    	\item zlib1g-dev   
    	\item libssl-dev
    	\item libldb-dev
    	\item libldap2-dev
    	\item libperl-dev
    	\item python-dev
    	\item libreadline-dev
    	\item libxml2-dev
    	\item libxslt1-dev
    	\item bison
    	\item flex
    	\item uuid-dev
    	\item make
    	\item gcc
    	\item libsystemd-dev
    	\item libxml2-utils
    	\item xsltproc
    \end{itemize}
    Install-Kommando: 
    
       
    sudo ./configure --prefix=/usr/local/pg12tde --with-openssl --with-perl --with-python --with-ldap --with-libxml --with-uuid=e2fs --with-systemd\\ \\
    Start-Kommando: 
    
    /usr/local/pg12tde/bin/initdb -D /media/db/cdw\_database/clinic\_instance
    

	\section{\ac{TDE}-Konfiguration} 
	
    \subsection{\ac{TDE}-Instanz-Dateien}
    Auf dem Betriebssystem wurde den Benutzer clinicuser angelegt, der dient die Administration der \ac{DB}-Instanz und besitzt keine administrative rechte auf dem Betriebssystem.
    Die Dateien der \ac{TDE}-Instanz befinden sich in dem Server unter /media/db/cdw\_database.
    \begin{itemize}
    	\item clinic\_instance -- Ordner der Instanz der \ac{cDWH} mit \ac{DB}- und Konfigurationsdateien.
    	%\item dbkeys -- Ordner der mit Schlüssel-Datei. Jede Schlüssel in der Datei hat 1000 Zufall-Zeichenketten, jede davon ist 1000 Charakter lang mit Buchstaben, Zahlen und Sonderzeichen, davon wird eine genommen, verschlüsselt und als Schlüssel für dem Start des PostgreSQL-\ac{TDE}-Services benützt.
    	\item sh\_scripts -- Shell-Skripts fürs Schlüssel-Manager.
    	\item dbBack -- Täglicher Backup der ganzen Instanz.
    \end{itemize}

    \subsection{Konfigurationsdateien}
    Die Datei postgresql.conf wurde wie folgt modifiziert:
    \begin{itemize}
    	\item port = 5433 \#Proxy der Instanz
    	\item listen\_addresses = '*' \# Maschinen auf denen die Instanz abrufbar ist
    	\item password\_encryption = scram-sha-256 \# Kennwort-Verschlüsselung Protokoll
    	\item encryption\_key\_command = '/media/db/cdw\_database/sh\_scripts/clinic\_instance\_key.sh'
    \end{itemize}
	In der Datei pg\_hba.conf wurden die Benutzer der Instanz definiert.
	\begin{itemize}
		\item local   all     all                      scram-sha-256 \# lokale Verbindungen
		\item host    all             all              0.0.0.0/0         scram-sha-256 \# externe Verbindungen
	\end{itemize}
