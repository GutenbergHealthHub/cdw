 \chapter{Struktur des \acs{csdwh}} \label{ch:strcsdwh} 
 
    \section{Beschreibung}
    
    Das \ac{csdwh} besitzt zwei strukturell gleiche \acp{db}, \texttt{staging} für die Produktion und \texttt{staging\_test} zum testen. Die \acp{db} sind in verschiedenen Schemata geteilt, jede davon entspricht eine Quelle oder Zusammenfassung von Systemen. Die Information der Schemata liegt in Kapitel~\ref{ch: schema}.
    
    \section{Liste der vorhandenen Schemata}
    
    \begin{center}
    	\captionof{table}{Schemata im \ac{csdwh}}
    	\begin{tabular}{|| l | p{8cm} ||} 
    		\hline
    		Schema & Information \\ 
    		\hline\hline
    		\texttt{centrallab} & Information aus dem Zentral Labor \\ 
    		\hline
    		\texttt{copra} & Information aus COPRA-System (PDMS) \\
    		\hline
    		\texttt{gtds} & Information aus dem \ac{gtds}  \\
    		\hline
    		\texttt{icd\_metadatainfo} & \ac{icd10gm}  \\
    		\hline
    		\texttt{kis} & Information aus dem \ac{kis}  \\
    		\hline
        	\texttt{metadata\_repository} & Metadata \\
        	\hline 
        	\texttt{ops\_metadatainfo} & \ac{ops}  \\
        	\hline
        	\texttt{p21} & Information aus \S 21  \\ 
        	\hline
        	\texttt{aktin} & Information des AKTIN-Projekts\\
        	\hline
        	\texttt{diz\_intern} & Administrative Information\\
        	\hline
        	\texttt{imagic} & Information aus iMagic \\
        	\hline
    	\end{tabular}
    \end{center}