\chapter{Diagnosen - Tabelle NDIA}
  Die Diagnosen der Fällen befinden sich in der Tabelle \texttt{NDIA}.

  \section{DQA\_NDIA\_DIAFP}

  \begin{table}[ht]
    \centering
    \caption{View DQA\_NDIA\_DIAFP}
    \label{tab:dqandiadiafp}
    \begin{tabular}{||l|l|p{10cm}||}
      \hline
      Spalte & Datentyp & Beschreibung \\ [0.5ex] \hline \hline
QUANTITY & bigint & Menge an Diagnosen mit einer bestimmten Fallpauschale \\ \hline
DIAFP & varchar & Fallpauschale (NULL bei nicht existierender Fallpauschale)\\ \hline
    \end{tabular}
  \end{table}
\clearpage 
  \section{DQA\_NDIA\_DIAGW}

  \begin{table}[ht]
    \centering
    \caption{View DQA\_NDIA\_DIAGW}
    \label{tab:dqandiadiagw}
    \begin{tabular}{||l|l|p{10cm}||}
      \hline
      Spalte & Datentyp & Beschreibung \\ [0.5ex] \hline \hline
QUANTITY & bigint & Menge an Diagnosen mit einer bestimmten Diagnostischen Gewissheit \\ \hline
DIAGW & varchar & DIAGW (NULL bei nicht existierender Diagnostische Gewissheit)\\ \hline
diag\_gewissheit & varchar & Klare Name der diagnostische Gewissheit (NULL bei nicht existierender Diagnostische Gewissheit)\\ \hline
    \end{tabular}
  \end{table}

  \section{DQA\_NDIA\_DIALO}

  \begin{table}[ht]
    \centering
    \caption{View DQA\_NDIA\_DIALO}
    \label{tab:dqandiadialo}
    \begin{tabular}{||l|l|p{10cm}||}
      \hline
      Spalte & Datentyp & Beschreibung \\ [0.5ex] \hline \hline
QUANTITY & bigint & Menge an Diagnosen mit einer bestimmten Lokalisation einer Diagnose \\ \hline
DIALO & varchar & Lokalisation einer Diagnose (NULL bei nicht existierender Lokalisation einer Diagnose)\\ \hline
localisation & varchar & Lokalisation einer Diagnose (NULL bei nicht existierende Lokalisation einer Diagnose)\\ \hline
    \end{tabular}
  \end{table}
 \clearpage
  \section{DQA\_NDIA\_DIAPR}

  \begin{table}[ht]
    \centering
    \caption{View DQA\_NDIA\_DIAPR}
    \label{tab:dqandiadiapr}
    \begin{tabular}{||l|l|p{10cm}||}
      \hline
      Spalte & Datentyp & Beschreibung \\ [0.5ex] \hline \hline
QUANTITY & bigint & Menge an Diagnosen mit einem bestimmten Kennzeichen Medizinische Nebendiagnose \\ \hline
DIAPR & varchar & Kennzeichen Medizinische Nebendiagnose (NULL bei nicht existierendem Kennzeichen Medizinische Nebendiagnose)\\ \hline
    \end{tabular}
  \end{table}

  \section{DQA\_NDIA\_DIASI}

  \begin{table}[ht]
    \centering
    \caption{View DQA\_NDIA\_DIASI}
    \label{tab:dqandiadiasi}
    \begin{tabular}{||l|l|p{10cm}||}
      \hline
      Spalte & Datentyp & Beschreibung \\ [0.5ex] \hline \hline
QUANTITY & bigint & Menge an Diagnosen mit einem bestimmten Grad der Sicherheit der Diagnose \\ \hline
DIASI & varchar & Grad der Sicherheit der Diagnose (NULL bei nicht existierendem Grad der Sicherheit der Diagnose)\\ \hline
    \end{tabular}
  \end{table}
 
  \section{DQA\_NDIA\_DIAZS}

  \begin{table}[ht]
    \centering
    \caption{View DQA\_NDIA\_DIAZS}
    \label{tab:dqandiadiazs}
    \begin{tabular}{||l|l|p{10cm}||}
      \hline
      Spalte & Datentyp & Beschreibung \\ [0.5ex] \hline \hline
QUANTITY & bigint & Menge an Diagnosen mit einem bestimmten Diagnosenzusatz \\ \hline
DIAZS & varchar & Diagnosenzusatz (NULL bei nicht existierendem Diagnosenzusatz)\\ \hline
diagnosenzusatz & varchar & Klare Name des Diagnosenzusatzes (NULL bei nicht existierendem zusatz)\\ \hline
    \end{tabular}
  \end{table}
 \clearpage
  \section{DQA\_NDIA\_DKAT1}

  \begin{table}[ht]
    \centering
    \caption{View DQA\_NDIA\_DKAT1}
    \label{tab:dqandiadkat1}
    \begin{tabular}{||l|l|p{10cm}||}
      \hline
      Spalte & Datentyp & Beschreibung \\ [0.5ex] \hline \hline
QUANTITY & bigint & Menge an Diagnosen mit einem bestimmten Identifikationsschlüssel für Diagnosekatalog \\ \hline
DKAT1 & varchar & Identifikationsschlüssel für Diagnosekatalog (NULL bei nicht existierendem Identifikationsschlüssel für Diagnosekatalog)\\ \hline
kapti & varchar & Kapitel bei der ICD10GM (NULL bei nicht existierendem Identifikationsschlüssel für Diagnosekatalog)\\ \hline
    \end{tabular}
  \end{table}

  \section{DQA\_NDIA\_DKAT2}

  \begin{table}[ht]
    \centering
    \caption{View DQA\_NDIA\_DKAT2}
    \label{tab:dqandiadkat2}
    \begin{tabular}{||l|l|p{10cm}||}
      \hline
      Spalte & Datentyp & Beschreibung \\ [0.5ex] \hline \hline
QUANTITY & bigint & Menge an Diagnosen mit einem bestimmten Identifikationsschlüssel für Diagnosekatalog \\ \hline
DKAT2 & varchar & Identifikationsschlüssel für Diagnosekatalog (NULL bei nicht existierendem Identifikationsschlüssel für Diagnosekatalog)\\ \hline
kapti & varchar & Kapitel bei der ICD10GM (NULL bei nicht existierendem Identifikationsschlüssel für Diagnosekatalog)\\ \hline
    \end{tabular}
  \end{table}
 \clearpage
  \section{DQA\_NDIA\_DKAT\_REF}

  \begin{table}[ht]
    \centering
    \caption{View DQA\_NDIA\_DKAT\_REF}
    \label{tab:dqandiadkatref}
    \begin{tabular}{||l|l|p{10cm}||}
      \hline
      Spalte & Datentyp & Beschreibung \\ [0.5ex] \hline \hline
QUANTITY & bigint & Menge an Diagnosen mit einer bestimmten Katalod-ID des Referenzkatalogs für Statistiken \\ \hline
DKAT\_REF & varchar & Katalod-ID des Referenzkatalogs für Statistiken (NULL bei nicht existierender Katalod-ID des Referenzkatalogs für Statistiken)\\ \hline
Jahr & text & Jahr des ICD10GM-Katalog (NULL bei nicht existierende Jahr)\\ \hline
    \end{tabular}
  \end{table}

  \section{DQA\_NDIA\_DKEY1}

  \begin{table}[ht]
    \centering
    \caption{View DQA\_NDIA\_DKEY1}
    \label{tab:dqandiadkey1}
    \begin{tabular}{||l|l|p{10cm}||}
      \hline
      Spalte & Datentyp & Beschreibung \\ [0.5ex] \hline \hline
QUANTITY & bigint & Menge an Diagnosen mit einer bestimmten Schlüsselung einer Diagnose \\ \hline
DKEY1 & varchar & Schlüsselung einer Diagnose (NULL bei nicht existierender Schlüsselung einer Diagnose)\\ \hline
icd10\_titel & varchar & Titel der ICD10GM der Diagnose (NULL bei nicht existierender Schlüsselung einer Diagnose)\\ \hline
    \end{tabular}
  \end{table}
 \clearpage
  \section{DQA\_NDIA\_DKEY2}

  \begin{table}[ht]
    \centering
    \caption{View DQA\_NDIA\_DKEY2}
    \label{tab:dqandiadkey2}
    \begin{tabular}{||l|l|p{10cm}||}
      \hline
      Spalte & Datentyp & Beschreibung \\ [0.5ex] \hline \hline
QUANTITY & bigint & Menge an Diagnosen mit einer bestimmten Schlüsselung einer Diagnose \\ \hline
DKEY2 & varchar & Schlüsselung einer Diagnose (NULL bei nicht existierender Schlüsselung einer Diagnose)\\ \hline
icd10\_titel & varchar & Titel der ICD10GM der Diagnose (NULL bei nicht existierender Schlüsselung einer Diagnose)\\ \hline
    \end{tabular}
  \end{table}

  \section{DQA\_NDIA\_DKEY\_REF}

  \begin{table}[ht]
    \centering
    \caption{View DQA\_NDIA\_DKEY\_REF}
    \label{tab:dqandiadkeyref}
    \begin{tabular}{||l|l|p{10cm}||}
      \hline
      Spalte & Datentyp & Beschreibung \\ [0.5ex] \hline \hline
QUANTITY & bigint & Menge an Diagnosen mit einer bestimmten Schlüsselung einer Referenzdiagnose für Statistiken \\ \hline
DKEY\_REF & varchar & Schlüsselung einer Referenzdiagnose für Statistiken (NULL bei nicht existierender Schlüsselung einer Referenzdiagnose für Statistiken)\\ \hline
icd10\_titel & varchar & Titel der ICD10GM der Diagnose (NULL bei nicht existierender Schlüsselung einer Referenzdiagnose für Statistiken)\\ \hline
    \end{tabular}
  \end{table}
 
 \clearpage
 
  \section{DQA\_NDIA\_DRG\_CATEGORY}

  \begin{table}[ht]
    \centering
    \caption{View DQA\_NDIA\_DRG\_CATEGORY}
    \label{tab:dqandiadrgcategory}
    \begin{tabular}{||l|l|p{10cm}||}
      \hline
      Spalte & Datentyp & Beschreibung \\ [0.5ex] \hline \hline
QUANTITY & bigint & Menge an Diagnosen mit einer bestimmten Kategorie einer DRG-Diagnose (Haupt- Neben) \\ \hline
DRG\_CATEGORY & varchar & Kategorie einer DRG-Diagnose (Haupt- Neben) (NULL bei nicht existierender Kategorie einer DRG-Diagnose (Haupt- Neben))\\ \hline
drg\_category & varchar & Name der Kategorie einer DRG-Diagnose (Haupt- Neben) (NULL bei nicht existierender Kategorie einer DRG-Diagnose (Haupt- Neben))\\ \hline
    \end{tabular}
  \end{table}

  \section{DQA\_NDIA\_DTYP1}

  \begin{table}[ht]
    \centering
    \caption{View DQA\_NDIA\_DTYP1}
    \label{tab:dqandiadtyp1}
    \begin{tabular}{||l|l|p{10cm}||}
      \hline
      Spalte & Datentyp & Beschreibung \\ [0.5ex] \hline \hline
QUANTITY & bigint & Menge an Diagnosen mit einem bestimmten Typ für ICD-10-Diagnosen \\ \hline
DTYP1 & varchar & DTYP1 (NULL bei nicht existierendem Typ für ICD-10-Diagnosen)\\ \hline
typ\_icd10 & varchar & Typ für ICD-10-Diagnosen (NULL bei nicht existierendem Typ für ICD-10-Diagnosen)\\ \hline
    \end{tabular}
  \end{table}
\clearpage
  \section{DQA\_NDIA\_DTYP2}

  \begin{table}[ht]
    \centering
    \caption{View DQA\_NDIA\_DTYP2}
    \label{tab:dqandiadtyp2}
    \begin{tabular}{||l|l|p{10cm}||}
      \hline
      Spalte & Datentyp & Beschreibung \\ [0.5ex] \hline \hline
QUANTITY & bigint & Menge an Diagnosen mit einem bestimmten Typ für ICD-10-Diagnosen \\ \hline
DTYP2 & varchar & Typ für ICD-10-Diagnosen (NULL bei nicht existierendem DTYP2)\\ \hline
typ\_icd10 & varchar & Typ für ICD-10-Diagnosen (NULL bei nicht existierendem Typ für ICD-10-Diagnosen)\\ \hline
    \end{tabular}
  \end{table}

  \section{DQA\_NDIA\_DTYPE\_REF}

  \begin{table}[ht]
    \centering
    \caption{View DQA\_NDIA\_DTYPE\_REF}
    \label{tab:dqandiadtyperef}
    \begin{tabular}{||l|l|p{10cm}||}
      \hline
      Spalte & Datentyp & Beschreibung \\ [0.5ex] \hline \hline
QUANTITY & bigint & Menge an Diagnosen mit einem bestimmten Typ für ICD-10-Diagnosen \\ \hline
DTYPE\_REF & varchar & Typ für ICD-10-Diagnosen (NULL bei nicht existierendem Typ für ICD-10-Diagnosen)\\ \hline
    \end{tabular}
  \end{table}

