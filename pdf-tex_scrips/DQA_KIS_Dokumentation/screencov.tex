\chapter{Screenings (COVID-19) - Tabelle /HSROM/SCREENCOV}
  Die Screenings für COVID-19 befinden sich in der Tabelle \texttt{HSROM/SCREENCOV}.

  \section{DQA\_/HSROM/SCREENCOV\_ \\ DISSCREENDLLENTRY}

  \begin{table}[ht]
    \centering
    \caption{View DQA\_/HSROM/SCREENCOV\_DISSCREENDLLENTRY}
    \label{tab:dqa/hsrom/screencovdisscreendllentry}
    \begin{tabular}{||l|l|p{10cm}||}
      \hline
      Spalte & Datentyp & Beschreibung \\ [0.5ex] \hline \hline
      QUANTITY & bigint & Menge an Screenings mit einem bestimmten i.s.h.med: Typ \\ \hline
      DISSCREENDLLENTRY & varchar & i.s.h.med: Typ (NULL bei nicht existierendem i.s.h.med: Typ)\\ \hline
    \end{tabular}
  \end{table}

 \clearpage

  \section{DQA\_/HSROM/SCREENCOV\_\\ DISSCREENRESULT}

  \begin{table}[ht]
    \centering
    \caption{View DQA\_/HSROM/SCREENCOV\_DISSCREENRESULT}
    \label{tab:dqa/hsrom/screencovdisscreenresult}
    \begin{tabular}{||l|l|p{10cm}||}
      \hline
      Spalte & Datentyp & Beschreibung \\ [0.5ex] \hline \hline
      QUANTITY & bigint & Menge an Screenings mit einem bestimmten i.s.h.med: Ergebnis \\ \hline
      DISSCREENRESULT & varchar & i.s.h.med: Ergebnis (NULL bei nicht existierendem i.s.h.med: Ergebnis)\\ \hline
    \end{tabular}
  \end{table}
 
  \section{DQA\_/HSROM/SCREENCOV\_\\ SCREENISO}

  \begin{table}[ht]
    \centering
    \caption{View DQA\_/HSROM/SCREENCOV\_SCREENISO}
    \label{tab:dqa/hsrom/screencovscreeniso}
    \begin{tabular}{||l|l|p{10cm}||}
      \hline
      Spalte & Datentyp & Beschreibung \\ [0.5ex] \hline \hline
      QUANTITY & bigint & Menge an Screenings mit einem bestimmten i.s.h.med: Screening: Isolation \\ \hline
      SCREENISO & varchar & i.s.h.med: Screening: Isolation (NULL bei nicht existierendem i.s.h.med: Screening: Isolation)\\ \hline
      hsrom\_screeniso & varchar & Name der Isolation (NULL bei nicht existierendem i.s.h.med: Screening: Isolation)\\ \hline
    \end{tabular}
  \end{table}

 \clearpage

  \section{DQA\_/HSROM/SCREENCOV\_\\ SCREENISOSTATE}

  \begin{table}[ht]
    \centering
    \caption{View DQA\_/HSROM/SCREENCOV\_SCREENISOSTATE}
    \label{tab:dqa/hsrom/screencovscreenisostate}
    \begin{tabular}{||l|l|p{10cm}||}
      \hline
      Spalte & Datentyp & Beschreibung \\ [0.5ex] \hline \hline
      QUANTITY & bigint & Menge an Screenings mit einem bestimmten i.s.h.med: Screening: Status Isolation \\ \hline
      SCREENISOSTATE & varchar & i.s.h.med: Screening: Status Isolation (NULL bei nicht existierendem i.s.h.med: Screening: Status Isolation)\\ \hline
    \end{tabular}
  \end{table}

  \section{DQA\_/HSROM/SCREENCOV\_\\ SCREENOPTION}

  \begin{table}[ht]
    \centering
    \caption{View DQA\_/HSROM/SCREENCOV\_SCREENOPTION}
    \label{tab:dqa/hsrom/screencovscreenoption}
    \begin{tabular}{||l|l|p{10cm}||}
      \hline
      Spalte & Datentyp & Beschreibung \\ [0.5ex] \hline \hline
      QUANTITY & bigint & Menge an Screenings mit eine bestimmten i.s.h.med: Screening: Testoptionen \\ \hline
      SCREENOPTION & varchar & i.s.h.med: Screening: Testoptionen (NULL bei nicht existierende i.s.h.med: Screening: Testoptionen)\\ \hline
    \end{tabular}
  \end{table}
  \clearpage
  \section{DQA\_/HSROM/SCREENCOV\_\\ SCREENOTHERS}

  \begin{table}[ht]
    \centering
    \caption{View DQA\_/HSROM/SCREENCOV\_SCREENOTHERS}
    \label{tab:dqa/hsrom/screencovscreenothers}
    \begin{tabular}{||l|l|p{10cm}||}
      \hline
      Spalte & Datentyp & Beschreibung \\ [0.5ex] \hline \hline
      QUANTITY & bigint & Menge an Screenings mit einem bestimmten i.s.h.med: Screening: Weitere/ Andere Kategorien \\ \hline
      SCREENOTHERS & varchar & i.s.h.med: Screening: Weitere/ Andere Kategorien (NULL bei nicht existierendem i.s.h.med: Screening: Weitere/ Andere Kategorien)\\ \hline
       hsrom\_screenothers & varchar & Name der weiteren/ anderen Kategorien (NULL bei nicht existierendem i.s.h.med: Screening: Weitere/ Andere Kategorien)\\ \hline
    \end{tabular}
  \end{table}

  \section{DQA\_/HSROM/SCREENCOV\_\\ SCREENSTATE}

  \begin{table}[ht]
    \centering
    \caption{View DQA\_/HSROM/SCREENCOV\_SCREENSTATE}
    \label{tab:dqa/hsrom/screencovscreenstate}
    \begin{tabular}{||l|l|p{10cm}||}
      \hline
      Spalte & Datentyp & Beschreibung \\ [0.5ex] \hline \hline
      QUANTITY & bigint & Menge an Screenings mit einem bestimmten i.s.h.med: Screening: Status \\ \hline
      SCREENSTATE & varchar & i.s.h.med: Screening: Status (NULL bei nicht existierendem i.s.h.med: Screening: Status)\\ \hline
      hsrom\_screenstate & varchar & Name des Status (NULL bei nicht existierendem i.s.h.med: Screening: Status)\\ \hline
    \end{tabular}
  \end{table}
  \clearpage
  \section{DQA\_/HSROM/SCREENCOV\_\\ SCREENTESTTYPE}

  \begin{table}[ht]
    \centering
    \caption{View DQA\_/HSROM/SCREENCOV\_SCREENTESTTYPE}
    \label{tab:dqa/hsrom/screencovscreentesttype}
    \begin{tabular}{||l|l|p{10cm}||}
      \hline
      Spalte & Datentyp & Beschreibung \\ [0.5ex] \hline \hline
      QUANTITY & bigint & Menge an Screenings mit einem bestimmten i.s.h.med: Screening: COVID-19 Testtyp Auswahl \\ \hline
      SCREENTESTTYPE & varchar & i.s.h.med: Screening: COVID-19 Testtyp Auswahl (NULL bei nicht existierendem i.s.h.med: Screening: COVID-19 Testtyp Auswahl)\\ \hline
      hsrom\_screentesttype & varchar & Name der COVID-19 Testtyp Auswahl (NULL bei nicht existierendem i.s.h.med: Screening: COVID-19 Testtyp Auswahl)\\ \hline
    \end{tabular}
  \end{table}
 
  \section{DQA\_/HSROM/SCREENCOV\_\\ SCREENVACCINE}

  \begin{table}[ht]
    \centering
    \caption{View DQA\_/HSROM/SCREENCOV\_SCREENVACCINE}
    \label{tab:dqa/hsrom/screencovscreenvaccine}
    \begin{tabular}{||l|l|p{10cm}||}
      \hline
      Spalte & Datentyp & Beschreibung \\ [0.5ex] \hline \hline
      QUANTITY & bigint & Menge an Screenings mit einem bestimmten i.s.h.med: Screening Impfstoff/ -präparat \\ \hline
      SCREENVACCINE & varchar & i.s.h.med: Screening Impfstoff/ -präparat (NULL bei nicht existierendem i.s.h.med: Screening Impfstoff/ -präparat)\\ \hline
       vaccine & varchar & Name des Impfstoffes/ -präparats (NULL bei nicht existierendem i.s.h.med: Screening Impfstoff/ -präparat)\\ \hline
        manufacturer & varchar & Name des Herstellers (NULL bei nicht existierendem i.s.h.med: Screening Impfstoff/ -präparat)\\ \hline
    \end{tabular}
  \end{table}

\clearpage

  \section{DQA\_/HSROM/SCREENCOV\_\\ SCREENVACCSTAT}

  \begin{table}[ht]
    \centering
    \caption{View DQA\_/HSROM/SCREENCOV\_SCREENVACCSTAT}
    \label{tab:dqa/hsrom/screencovscreenvaccstat}
    \begin{tabular}{||l|l|p{10cm}||}
      \hline
      Spalte & Datentyp & Beschreibung \\ [0.5ex] \hline \hline
      QUANTITY & bigint & Menge an Screenings mit einem bestimmten i.s.h.med: Screening Impfstatus (geimpft: ja/nein) \\ \hline
      SCREENVACCSTAT & varchar & i.s.h.med: Screening Impfstatus (geimpft: ja/nein) (NULL bei nicht existierendem i.s.h.med: Screening Impfstatus (geimpft: ja/nein))\\ \hline
      hsrom\_screenvaccstat & varchar & Name des Impfstatus (geimpft: ja/nein) (NULL bei nicht existierendem i.s.h.med: Screening Impfstatus (geimpft: ja/nein))\\ \hline
    \end{tabular}
  \end{table}
 \clearpage
